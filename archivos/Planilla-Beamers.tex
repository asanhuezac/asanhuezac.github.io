\documentclass[12pt]{beamer}
\AtBeginSection[]{
\begin{frame}{}
\tableofcontents[currentsection]
\end{frame}
}
\mode<presentation>{\usetheme{Antibes}}
\title{Título} 
\author{Autores / Ramo \\
\vspace{0.1cm}
Texto1 \\
Texto2}
\date{Fecha}

\begin{document}
\begin{frame}{}
    \maketitle
\end{frame}
\begin{frame}{Correos}
\begin{columns}
\column{0.5\textwidth}
\begin{itemize}
    \item \href{mailto:}{mail 1}
    \item \href{mailto:}{mail 2}

\end{itemize}

\end{columns}
\end{frame}

\begin{frame}{Contenidos de Presentación}
    \tableofcontents[] % Aquí se despliegan cada una de las \section{} que generemos
\end{frame}

\section{Sección 1}
\begin{frame}{Algo de la sección 1}
\onslide<1->{} % Este comando nos permite poner contenido de manera pulatina en una slide/frame
\vspace{5mm}
\onslide<2->{ \\
\textbf{Respuesta:}\\
}
\end{frame}

\section{Sección 2}

\begin{frame}{Algo de la sección 2}
\onslide<1->{Primer Texto reproducido de la slide} \\
\onslide<2->{Segundo Texto reproducido de la slide}

\end{frame}

\begin{frame}
Otra cosa de la sección 2 sin header
\begin{center}
\includegraphics[scale=0.5]{alguna_imagen}    
\end{center}
\end{frame}



\end{document}
